\section{Introduction}


The great success of the real-time communication has been witnessed in
the last decade. The popular software, like Skype, QQ and Apple's
Facetime, make the real-time communication as a cheaper alternative to
the calls. Nowadays, with the emerging mobile market, these
applications were migrated to the mobile platform as well. However,
like the other mobile applications, the power issue is hindering the
deployment of real-time communication on the mobile
devices. Especially in the real-time communication app, several
power-hungry modules, like wireless network, camera and etc. So
maximizing the power savings is desired and attractive in the
real-time communication.

%% The real-time communication, as a considerably
%% cheaper alternative to the calls, makes QQ, Skype, Apple's
%% Facetime and etc. so popular all round the world. More attractively,
%% with the bandwidth growth of the Internet, the real-time communication
%% is always equipped with the video channel so that two ends can talk
%% face-to-face. Now the prosperity of mobile devices, including
%% smartphones, tablets and emerging wearable devices, offers another
%% available platform for the real-time communication. The real-time
%% communication is more desired on these portable devices, which can
%% benefit from the wireless networks everywhere. However, new challenges
%% to deliver video stream on the mobile devices have raised and one of
%% them is to maximize the power savings.

According to the previous research ~\cite{AG10}, about 38\% to 68\% of the
total energy is consumed by the display subsystem during the video
playback, and this portion is growing on the new generation of
devices, which are equipped with more powerful GPUs and more
large-sized display panels. We also expected the average power
consumption during the real-time communication will be cut down if the
display subsystem is optimized. One existing technique, named
backlight scaling, can be used to reduce the energy consumption of
Liquid-Crystal Displays (LCD). The power consumption of LCD is
dominated by the backlight intensity, so this technique tries to dim
the backlight as much as possible while enhancing the luminance of the
pixels, which are rendered on the screen, at the same time. In this
way, the modified images on the screen are perceived as the original
ones by the human eye.  


In this paper, we explored the possibility of deploying the backlight
scaling technique with the real-time communication. The backlight
scaling technique is composed of three stages. 1) Histogram generation
stage. 2) Backlight levels determination stage. 3) Pixels compensation
stage. The first stage and the last stage are both computation
intensive tasks when performing these operations on each frame. In the
second stage, previous research applies the dynamic programming to
find the globally optimal backlight levels without violating some
constraints. Not only is the extra CPU time required, but also the
backlight levels can be determined on-the-fly. All of these obstruct
the backlight scaling technique being applied in the real-time
communication.

In the real-time communication, like the video conferencing, the video
contents are generated from multiple sources. Even for the one-to-one
video call, double frames are generated other than the common Internet
streaming, like watching the movies. To avoid processing the same
frame repeatedly, we migrate the histogram generation stage to the
sender side. We put the luminance information inside each frame
without building an additional channel. Also duplicated information is
hidden for the possible loss during the video encoding/decoding and
the network transmission.

The dynamic programming algorithm is necessary in the common video
playback scenario since there are some constraints we have to conform
to. The backlight variations must be limited when the flickering
effect is not expected in the user experience. Also the backlight
should not be scaled down too much to cause the distortion. However,
the dynamic programming is not applicable in real-time communication
due to the dynamically generating video contents. Even if this
algorithm is partially applied, significant delays are introduced in
the sessions. Fortunately, in the real-time communication, we can risk
scaling the backlight levels in greedy manner because no scenes
switching are expected. 

When the GPU is commonly equipped in the mobile devices, we are able
to take advantage of this facility to offload some tasks from the
CPU. For the purpose of rendering the received frames in timely
manner, we use the OpenGL ES 2.0 shaders to perform the pixels
enhancement stage. Little delays are introduced in this way. Moreover,
the power savings won't be offset by enabling GPU since GPU is widely
used in the video conferencing for composing frames together and then
rendering them.

This paper is organized as follows: the background is introduced and
some related work is reviewed in Section \ref{sec:background}. We propose
our design in Section \ref{sec:design}. The details of implementation
are illustrated in Section \ref{sec:implementation}. Then we discuss
the evaluation in Section \ref{sec:evaluation} and conclude our work
in Section \ref{sec:conclusion}.

% stress the importance real-time communication on mobile devices
% This part is redundant, make it concise.
% Don't discuss the details of WebRTC. Only mention the general idea
% of Real-time communication.


%% On the other hand, Dynamic Adaptive Streaming over
%% HTTP(DASH) makes a salient example that how multimedia streaming is
%% delivered nowadays. WebRTC(Web Real-Time Communication), technique
%% that integrating real-time communication over HTTP, is standardized by
%% W3C and the IETF. Thus a cross-platform solution will be handy and the
%% growth of real-time communication on new platform is
%% expected.
%% As an advanced functionality, video call becomes an essential
%% component in many *communication applications/wechat, skype... check
%% in twin cloud paper*. The needs for unifying the interface is
%% increasingly desirable. The success of DATH *add name* hints people
%% the importance of HTTP in video/audio streaming scenarios. WebRTC as a
%% new API existing in HTML5 drops in developer's sight. Mainstream
%% browsers integrates the function into there production, i.e. Chrome,
%% Opera, Firefox. Developers can easily include video call in their page
%% via few lines of javascript code.


%% In this paper, we explored the possibility if an effective LCD
%% power-saving technique, named luminance compensation, can be applied
%% on the real-time communication services. There are three steps in this
%% technique. 1) Extract the luminance histrogram of the frames. 2)
%% Decide the backlight levels and do scaling when the frames are
%% rendered. 3) According to the backlight levels, enhance the luminance
%% component of the pixels. This technique has been successfully adopted
%% on some other streaming services, however, there are still 
%% several challenges in the real-time case:
%% \begin{itemize}
%%   \item
%%     {
%%       Unlike the other streaming services of predefined content, in
%%       real-time communication, there is no way to build a utility to
%%       extract and analyze the videos in the offline manner. The extraction
%%       process, which is computation intensive, has to be installed
%%       along with the services. 
%%     }
%%   \item
%%     {
%%       In a N-party communication, $N$ frames are rendered on one screen
%%       at the same time. $N$ times the input data makes the extraction
%%       operation even worse.
%%     }
%%   \item
%%     {
%%       To decide the backlights levels, a dynamic programming algorithm
%%       has been proposed. However, this handy soloution requires large
%%       buffer size to collect enough luminance information and then to
%%       reach the global optimal result. However, this is not practical
%%       in the real-time case.
%%     }
%% \end{itemize}

%% designed a system composed three components in the names of {\bf
%%   Scanning Module}, {\bf Adjustment Module} and {\bf Rendering
%%   Module}. These components are integrated into real-time
%% communication.


% conventional
%% On the other hand, people doesn't satisfy themselves with traditional
%% phone call, which make them only listen to their relatives. On
%% smartphone, installing communication applications becomes available
%% and people can also see the other end over one call. *not only on the
%% traditional PC*. Moreover, video call is much cheaper, sometimes even
%% free. In the future day, we can easily make video call over webRTC,
%% the unifying interface in browsers. Since both on the mobile and PC we
%% can access it via browsers. Even more, Google has been making effort
%% on packaging their implementation in Chrome as available for Android
%% App developers. So we can make such call via the simple thing.

%% The power-saving scheme for VoIP is few, since the most applications
%% are properties of company. Researchers is hard to figure out which
%% part is enable to save the energy. Also the old applications are
%% designed for PC client. The achievement is in the network layer,
%% adapting the transmission protocols to specific scenarios *I guess,
%% check it*. On the other hand, many effective energy effective schemes
%% were proposed for streaming on mobile platform, such as *give some
%% example, ref to our paper*. This motivate us to put these schemes in
%% the real-time multimedia case.

%% Thanks to Google, the Chromes is an open source project, in the nature
%% of things the WebRTC is part of it as libjingle*introduce to
%% libjingle*. More fortunately, the libjingle is able to be integrated
%% as 3rd library with Android applications. So the Android application
%% can take advantage of this aside web browsers.

%% Display is the most great part of mobile power consuming. As
%% LCD*introduce the LCD?*, one 
%% efficient way to save power is adjust the backlight with compensating
%% the pixels. there are works done on the local video case and online
%% streaming case. Usually the GPU is used to dynamically scan frames and
%% retrieve the highest pixels. we have some opportunities:
%% \begin{itemize}
%%   \item
%%     {
%%       % Challenge: Cooperation
%%       In the real-time case, or the p2p case, video is generated via
%%       clients. We don't need to passively accept the frame and parse
%%       frames until them are decoded. On the other hand, we can scan
%%       the frames before encoding when the stream is
%%       generated. How can we effectively do this? one challenge is how
%%       we pass the information to the other side. we
%%       don't need to open another data channel on this. put the max
%%       pixel at the corner.
%%       *We can also split the scanning process*
%%       we offload the DP part to the receiver.
%%     }
%%   \item
%%     {
%%       % Challenge: take advantage of the scenarios
%%       Also the real-time communication has its characteristics. The
%%       frames staying relative still. We can adjust the max liminance
%%       in greedy manner. More aggressively, we can adjust the frames
%%       in case of skipping some frames. 
%%     }
%%   \item
%%     {
%%       % Challenge: graphical composition
%%       The receiver is asked to compose frames received from several
%%       senders. 
%%     }
%%   \item
%%     {
%%       % The DP problem
%%       The DP is not optimal if the power-backlight is not linear
%%       relationship. But the Greedy is even more not optimal.

%%       In real-communication users is more likely to be tolerant. 
%%     }
%% \end{itemize}


