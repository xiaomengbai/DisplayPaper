\section{Introduction}

% stress the importance real-time communication on mobile devices
% This part is redundant, make it concise.
As an advanced functionality, video call becomes an essential
component in many *communication applications/wechat, skype... check
in twin cloud paper*. The needs for unifying the interface is
increasingly desirable. The success of DATH *add name* hints people
the importance of HTTP in video/audio streaming scenarios. WebRTC as a
new API existing in HTML5 drops in developer's sight. Mainstream
browsers integrates the function into there production, i.e. Chrome,
Opera, Firefox. Developers can easily include video call in their page
via few lines of javascript code.

% conventional 
On the other hand, people doesn't satisfy themselves with traditional
phone call, which make them only listen to their relatives. On
smartphone, installing communication applications becomes available
and people can also see the other end over one call. *not only on the
traditional PC*. Moreover, video call is much cheaper, sometimes even
free. In the future day, we can easily make video call over webRTC,
the unifying interface in browsers. Since both on the mobile and PC we
can access it via browsers. Even more, Google has been making effort
on packaging their implementation in Chrome as available for Android
App developers. So we can make such call via the simple thing.


The power-saving scheme for VoIP is few, since the most applications
are properties of company. Researchers is hard to figure out which
part is enable to save the energy. Also the old applications are
designed for PC client. The achievement is in the network layer,
adapting the transmission protocols to specific scenarios *I guess,
check it*. On the other hand, many effective energy effective schemes
were proposed for streaming on mobile platform, such as *give some
example, ref to our paper*. This motivate us to put these schemes in
the real-time multimedia case.

Thanks to Google, the Chromes is an open source project, in the nature
of things the WebRTC is part of it as libjingle*introduce to
libjingle*. More fortunately, the libjingle is able to be integrated
as 3rd library with Android applications. So the Android application
can take advantage of this aside web browsers.

Display is the most great part of mobile power consuming. As
LCD*introduce the LCD?*, one 
efficient way to save power is adjust the backlight with compensating
the pixels. there are works done on the local video case and online
streaming case. Usually the GPU is used to dynamically scan frames and
retrieve the highest pixels. we have some opportunities:
\begin{itemize}
  \item
    {
      % Challenge: Cooperation
      In the real-time case, or the p2p case, video is generated via
      clients. We don't need to passively accept the frame and parse
      frames until them are decoded. On the other hand, we can scan
      the frames before encoding when the stream is
      generated. How can we effectively do this? one challenge is how
      we pass the information to the other side. we
      don't need to open another data channel on this. put the max
      pixel at the corner.
      *We can also split the scanning process*
      we offload the DP part to the receiver.
    }
  \item
    {
        % Challenge: take advantage of the scenarios
        Also the real-time communication has its characteristics. The
        frames staying relative still. We can adjust the max liminance
        in greedy manner. More aggressively, we can adjust the frames
        in case of skipping some frames. 
    }
\end{itemize}

the paper is organized as follows: *filled this*

