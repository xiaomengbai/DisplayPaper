\section{Introduction}

% stress the importance real-time communication on mobile devices
% This part is redundant, make it concise.
% Don't discuss the details of WebRTC. Only mention the general idea
% of Real-time communication.

We have witnessed the great success of real-time communication in the
past decade. Skype, QQ and Apple's Facetime are honored all around the
world, as the result of that people can contact anyone only if both of
them are online. This service provides a much cheaper alternative
other than the telephone. More attractively, with the bandwidth growth
of Internet, real-time communication is always equipped video channel
that two ends can talk face-to-face. Now the prosperity of mobile
devices, including smart phones, tablets and emerging wearable
devices, offers another available platform for the real-time
communication. On the other hand, Dynamic Adaptive Streaming over
HTTP(DASH) makes a salient example that how multimedia streaming is
delivered nowadays. WebRTC(Web Real-Time Communication), technique
that integrating real-time communication over HTTP, is standardized by
W3C and the IETF. Thus a cross-platform solution will be handy and the
growth of real-time communication on new platform is
expected. However, new challenges are found on mobile devices when
delivering streaming. One of them is power saving. 
%% As an advanced functionality, video call becomes an essential
%% component in many *communication applications/wechat, skype... check
%% in twin cloud paper*. The needs for unifying the interface is
%% increasingly desirable. The success of DATH *add name* hints people
%% the importance of HTTP in video/audio streaming scenarios. WebRTC as a
%% new API existing in HTML5 drops in developer's sight. Mainstream
%% browsers integrates the function into there production, i.e. Chrome,
%% Opera, Firefox. Developers can easily include video call in their page
%% via few lines of javascript code.


On the mobile platform, about 38\% to 68\% of total power is consumed
by the display subsystem during the video playback, while this portion
is expected to ever grow on the new generation of devices, which are
equipped more powerful GPUs and more large-sized display panels. We
believe that the real-time communication services will be more
attractive if the power consumption of display subsystem is cut
down. Since the Liquid-Crystal Displays(LCD) is dominant in current
transparent display market, in this paper, we are about to propose a
scheme can help reduce the power consumption of LCD. 


In this paper, we explored the possibility if we can apply an
effective LCD power-saving technique, named luminance compensation, on
the real-time communication services. This technique has been adopted
on the Internet streaming services, while in the real-time case,
several challenges can be found:
\begin{itemize}
  \item
    {
      Unlike the Internet streaming services, the streaming content of
      real-time content is not predefined. This make extracting pixel
      information, which is computation intensive, more difficult.
    }
  \item
    {
      In real-time communication, a common output frame is composed of
      $N$ frames while there are $N$ participants. This makes
      extracting operation more burdensome.
    }
  \item
    {
      While the pixels extracting operations and backlight adjustment
      procedure both consumes extra CPU/GPU resource, This may offset
      the savings. It is alluring to make these operations more
      power-effective.
    }
\end{itemize}

%% designed a system composed three components in the names of {\bf
%%   Scanning Module}, {\bf Adjustment Module} and {\bf Rendering
%%   Module}. These components are integrated into real-time
%% communication.


% conventional
%% On the other hand, people doesn't satisfy themselves with traditional
%% phone call, which make them only listen to their relatives. On
%% smartphone, installing communication applications becomes available
%% and people can also see the other end over one call. *not only on the
%% traditional PC*. Moreover, video call is much cheaper, sometimes even
%% free. In the future day, we can easily make video call over webRTC,
%% the unifying interface in browsers. Since both on the mobile and PC we
%% can access it via browsers. Even more, Google has been making effort
%% on packaging their implementation in Chrome as available for Android
%% App developers. So we can make such call via the simple thing.

%% The power-saving scheme for VoIP is few, since the most applications
%% are properties of company. Researchers is hard to figure out which
%% part is enable to save the energy. Also the old applications are
%% designed for PC client. The achievement is in the network layer,
%% adapting the transmission protocols to specific scenarios *I guess,
%% check it*. On the other hand, many effective energy effective schemes
%% were proposed for streaming on mobile platform, such as *give some
%% example, ref to our paper*. This motivate us to put these schemes in
%% the real-time multimedia case.

%% Thanks to Google, the Chromes is an open source project, in the nature
%% of things the WebRTC is part of it as libjingle*introduce to
%% libjingle*. More fortunately, the libjingle is able to be integrated
%% as 3rd library with Android applications. So the Android application
%% can take advantage of this aside web browsers.

%% Display is the most great part of mobile power consuming. As
%% LCD*introduce the LCD?*, one 
%% efficient way to save power is adjust the backlight with compensating
%% the pixels. there are works done on the local video case and online
%% streaming case. Usually the GPU is used to dynamically scan frames and
%% retrieve the highest pixels. we have some opportunities:
%% \begin{itemize}
%%   \item
%%     {
%%       % Challenge: Cooperation
%%       In the real-time case, or the p2p case, video is generated via
%%       clients. We don't need to passively accept the frame and parse
%%       frames until them are decoded. On the other hand, we can scan
%%       the frames before encoding when the stream is
%%       generated. How can we effectively do this? one challenge is how
%%       we pass the information to the other side. we
%%       don't need to open another data channel on this. put the max
%%       pixel at the corner.
%%       *We can also split the scanning process*
%%       we offload the DP part to the receiver.
%%     }
%%   \item
%%     {
%%       % Challenge: take advantage of the scenarios
%%       Also the real-time communication has its characteristics. The
%%       frames staying relative still. We can adjust the max liminance
%%       in greedy manner. More aggressively, we can adjust the frames
%%       in case of skipping some frames. 
%%     }
%%   \item
%%     {
%%       % Challenge: graphical composition
%%       The receiver is asked to compose frames received from several
%%       senders. 
%%     }
%%   \item
%%     {
%%       % The DP problem
%%       The DP is not optimal if the power-backlight is not linear
%%       relationship. But the Greedy is even more not optimal.

%%       In real-communication users is more likely to be tolerant. 
%%     }
%% \end{itemize}

the paper is organized as follows: *filled this*

