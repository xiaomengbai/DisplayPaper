\begin{abstract}

The display subsystem of a mobile device usually consumes 30\%-60\%
of the total battery power in video streaming. Therefore, a few designs have aimed to
reduce the display power consumption.  The
basic idea is to dim the backlight level while properly compensating
the pixel luminance to maintain image fidelity.  The luminance
compensation and proper luminance calculation are computation
intensive and demand per-frame luminance information. For these reasons, existing
schemes only work for video-on-demand  where 
each frame (and thus the luminance information) is available in advance. In addition, they demand
additional computing resource support. Otherwise, if the computation
is conducted on the mobile device, the power consumption due to such computing
can easily offset the power saving by dimming the backlight.  

In this work, we set to
investigate power saving for video calls on mobile devices.  Different
from video-on-demand, real-time video calls are highly delay
sensitive and the frame luminance information is not known in
advance. Moreover, video calls often involve multiple
streaming sources from multiple ($\ge$2) participants, making it more
difficult.  Because there are fewer background changes and a often
slower frame rate in video calls, we design a Greedy Display Power
saving scheme, called LCD-GDP, which utilizes the commonly available GPU on mobile
devices without demanding additional support.  Our design is
implemented on WebRTC, a popular real-time web browser based video
call standard.  Experiments show that our scheme can save upto 33\%
power consumption in video calls without affecting the video call quality.

%% Power saving is a significant issue at the mobile platform. Various
%% techniques were proposed to extend the playback time of the
%% videos. Backlight scaling is one of them and target the display
%% subsystem. However, the multimedia content need to be delivered in a
%% timely manner in the real-time communication. This raises challenges
%% towards the existing backlight scaling technique. The backlight
%% scaling technique is composed of computation intensive tasks,
%% including the luminance histogram generation and the pixels
%% luminance compensation on every frame. Multiple streaming sources in
%% the video conference make this problem more difficult. Moreover,
%% global luminance information is necessary for finding the optimal
%% backlight levels across the streaming. This is clearly impossible in
%% the real-time case. In this paper, we propose a greedy algorithm
%% determining the backlight level of current frame only from the last
%% one. We also take advantage of the GPU to relive the burden of the
%% CPU on compensating the luminance of pixels. We integrate our
%% prototype in the AppRTC, an Android app based on WebRTC, and find
%% that our scheme can save up to 39.8\% energy during one
%% communication session.
\end{abstract}
