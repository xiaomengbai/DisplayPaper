\section{evaluation}
\label{sec:evaluation}

In worrying about our scheme will make the quality of the video calls
degrades, we measured the quality metrics in fps, bitrate and latency
of our greedy scheme compared to the original version. All the results
is reported by the webRTC app itself. In measurement, we found the
quality of the video in webRTC is self-adjusted. There are two factors
affecting the fps and bitrate: the average luminance of the frames and
the content of the frames. The webRTC app intends to reduce the
fps/bitrate in darker images. Also the static frames gains higher
fps/bitrate other than the frames containing moving objects. To
evaluate how much overhead is introduced by our scheme, we only use
the static scenario as the test case. We expect if our greedy
algorithm won't impair the visual quality of the static scenario,
which features higher fps and bitrate, the quality of scenarios
containing moving objects will not be damaged as well.


Two devices are used in this measurement, one Samsung 10.1-inch tablet
and one NEXUS 4 smartphone. The data is collected from the tablet. We test the
real-time communication in the resolution of $640$x$480$, and set the
upperbound of the frames generation by the camera as $30$. Both the
dark scenario and the bright scenario are measured. We maintained the
communication at least $5$ minutes. We test the {\it loopback} and the
{\it remote} respectively. The {\it loopback} means the tablet
communicate to itself via network. And the {\it remote} means
communication between two devices.

\begin{table}[h]
  \centering
  \caption{video quality}
  \label{tab:quality}
  %% fps input | fps sent | fps received | fps decoded | fps output | sent bitrate | receive bitrate
  \begin{tabular}{|c|c|c|c|c|c|c|c|c|} %lcr
    \hline
    & \multicolumn{6}{|c|}{fps} & \multicolumn{2}{|c|}{bitrate} \\ \hline
    & \multicolumn{2}{|c|}{input} & \multicolumn{2}{|c|}{sent}
    & \multicolumn{2}{|c|}{output} & sent & recv
    \\ \hline
    & $E$ & $\delta$ & $E$ & $\delta$ & $E$ & $\delta$ & & \\ \hline
    \multicolumn{9}{|c|}{bright scenario/loopback} \\ \hline
    original & $24.67$ & $1.01$ & $22.73$ & $3.59$ & $15.37$ & $2.44$ &
    $81076$ & $81344$ \\ \hline
    greedy & $24.51$ & $1.06$ & $21.64$ & $2.76$ & $15.19$ & $2.06$ &
    $44832$ & $45193$ \\ \hline
    \multicolumn{9}{|c|}{bright scenario/remote} \\ \hline
    original & $23.78$ & $1.66$ & $19.09$ & $2.81$ & $13.91$ & $4.41$ &
    $269094$ & $41583$ \\ \hline
    greedy & $23.99$ & $1.18$ & $18.79$ & $3.06$ & $13.60$ & $3.17$ &
    $267309$ & $32905$ \\ \hline
    \multicolumn{9}{|c|}{dark scenario/loopback} \\ \hline
    original & $7.86$ & $0.35$ & $8.00$ & $0.00$ & $7.87$ & $0.40$ &
    $168306$ & $168306$ \\ \hline
    greedy & $7.88$ & $0.39$ & $8.00$ & $0.05$ & $7.87$ & $0.39$ &
    $165936$ & $165936$ \\ \hline
    \multicolumn{9}{|c|}{dark scenario/remote} \\ \hline
    original & $8.02$ & $1.37$ & $8.00$ & $0.42$ & $4.85$ & $0.72$ &
    $462505$ & $924722$ \\ \hline
    greedy & $7.85$ & $0.48$ & $8.00$ & $0.18$ & $4.96$ & $0.53$ &
    $246837$ & $779452$ \\ \hline
    %% & \multicolumn{2}{inputfps} & \multicolumn{2}{sentfps} & \multicolumn{2}{outputfps} & sentbitrate & recvbitrate \\ \hline
  \end{tabular}
  
\end{table}

From the Table~\ref{tab:quality}, we can find that our greedy scheme
doesn't affect the fps. 

%% \begin{table}[h]
%%   \centering
%%   \caption{power consumption}
%%   \label{tab:power_consumption}
%%   %% device | dark | bright |
%%   \begin{tabular}{|c|c|c|c|c|} %lcr
%%     \hline
%%     & \multicolumn{2}{|c|}{SAMSUNG(dark)} & \multicolumn{2}{|c|}{SAMSUNG(bright)} \\ \hline
%%     & no-adaption & with adaption & no-adaption & with adaption \\ \hline
%%     NEXUS(dark) & $\sim{6250}$ mW & $\sim{3810}$ mW & $\sim{6900}$ mW & $\sim{4150}$ mW  \\ \hline
%%     NEXUS(bright) & $\sim{6500}$ mW & $\sim{6400}$ mW & $\sim{7250}$ mW & $\sim{6800}$ mW \\ \hline
%%   \end{tabular}
  
%% \end{table}


%% Expr 1:
%% Comparison between normal case and cases applied our scheme.
%% Prove that our implementation doesn't downgrade the user experience.
%% \begin{itemize}
%%   \item{bitrate}
%%   \item{fps}
%%   \item{maybe others}
%% \end{itemize}

Expr 2:
Compare the distortion of frames among 1) normal case 2) DP 3)
Greedy. store the frames under different brightness. (90\%-100\%,
70\%-80\%, 50\%-60\%).
To align the frames. (Can try index the frames in several pixels)
\begin{itemize}
  \item{PSNR}
  \item{SSIM}
\end{itemize}


\subsection{Power Consumption}
We conducted a series of simple experiments to evaluate the power
consumption of our system. Two devices, one NEXUS 4 smartphone and one
Samsung Galaxy 10 inch tablet, were used in these experiments. We
installed the original version WebRTC app and an upgraded version
equipped with our dynamic programming scheme. When these two devices
connected to each other, we put them under different environments to
evaluate if our scheme could outperform the original version. Two
extreme environments were selected purposely. One was an extreme
bright environment that our scheme didn't work any more. Under the
other environment, which is an extreme dark case, the backlight could
always be scaled to the lowest threshold($0.4$). We measured the power
consumption of the tablets in all of these cases and the results are
shown in ~\ref{tab:power_consumption}.


\begin{table}[h]
  \centering
  \caption{power consumption}
  \label{tab:power_consumption}
  %% device | dark | bright |
  \begin{tabular}{|c|c|c|c|c|} %lcr
    \hline
    & \multicolumn{2}{|c|}{SAMSUNG(dark)} & \multicolumn{2}{|c|}{SAMSUNG(bright)} \\ \hline
    & no-adaption & with adaption & no-adaption & with adaption \\ \hline
    NEXUS(dark) & $\sim{6250}$ mW & $\sim{3810}$ mW & $\sim{6900}$ mW & $\sim{4150}$ mW  \\ \hline
    NEXUS(bright) & $\sim{6500}$ mW & $\sim{6400}$ mW & $\sim{7250}$ mW & $\sim{6800}$ mW \\ \hline
  \end{tabular}
  
\end{table}

From the table, we found that the original version WebRTC app consumed
similar power, which was from $6250$ mA to $7250$ mA, under different
environments. The slight difference are caused by the mutable fps. The
WebRTC app actively reduces the fps if it detects that the environment
gets dark. So the power consumption was found lowest when both the
devices were put in the dark environment. The most observable power
savings were found in the case that the NEXUS was put in the dark
environment. The dimmed frames sent by the NEXUS help the tablet take
advantage of the luminance compensation technique.  Notice that
although our scheme outperformed the original version in the both
bright case. This power reduction comes from the lower fps. The
dynamic programming algorithm, holding too many frames before
delivering them to the rendering module, will definitely degrade the
video quality.
%% The
%% dynamic programming algorithm has to hold enough frames to adjust
%% backlight, so that it can not deliver the frames to the rendering
%% module in time.


